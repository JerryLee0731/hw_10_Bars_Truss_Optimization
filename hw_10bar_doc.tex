\documentclass[12pt, a4paper]{article}

% \usepackage{extsizes}  % 字體文件使用 14pt
\usepackage{xeCJK}
\usepackage[margin=2.0cm]{geometry}
\usepackage{fancyhdr}
\usepackage{graphicx}
\usepackage{amsmath, amssymb, amsfonts}
\usepackage{listings}  % code showing
\usepackage{pagecolor}  % code color define
\usepackage{tocloft}  % contents
\usepackage{hyperref} % 超連結

% -- 字體 (注意檔案是否存在)
\setCJKmainfont{DFKai-SB}
\setmainfont{Times New Roman}
% \setCJKmainfont[
    % Path="path to folder"
    % Path="/home/ych/Repository/solab-latex/Template/Thesis/fonts/"
% ]{kaiu}
% \setmainfont[
    % Path="path to folder",
    % Path="/home/ych/Repository/solab-latex/Template/Thesis/fonts/",
    % BoldFont={Times New Roman Bold},
    % ItalicFont={Times New Roman Italic},
    % BoldItalicFont={Times New Roman Bold Italic}
% ]{Times New Roman}
% \setsansfont{Arial}
% \setmonofont{DroidSansMono Nerd Font}

% -- 格式
\pagestyle{fancy}  % 使用 header
\fancyhf{}
\setcounter{secnumdepth}{-1}  % 移除 section 數字
\renewcommand{\cftsecleader}{\cftdotfill{\cftdotsep}}

%% -- 標題使用中文取代
\renewcommand{\figurename}{圖}
\renewcommand{\tablename}{表}
\renewcommand{\lstlistingname}{程式碼}

%% -- 間距設定
% \linespread{1.2}\selectfont  % 行距
\setlength{\headheight}{29pt}  % header 最低高度
\setlength{\parindent}{0pt}  % 取消縮排
\setlength{\parskip}{0.5em}  % 段落距離
\setlength{\abovecaptionskip}{10pt}  % 圖表標題 caption 與圖表的距離
\setlength{\belowcaptionskip}{10pt}

%% -- 顯示程式碼
\definecolor{codegreen}{rgb}{0,0.6,0}
\definecolor{codegray}{rgb}{0.5,0.5,0.5}
\definecolor{codepurple}{rgb}{0.58,0,0.82}
\definecolor{backcolour}{rgb}{0.95,0.95,0.92}
\lstdefinestyle{pystyle}{
    backgroundcolor=\color{backcolour},
    commentstyle=\color{codegreen},
    keywordstyle=\color{magenta},
    numberstyle=\footnotesize\color{codegray},
    stringstyle=\color{codepurple},
    basicstyle=\ttfamily\footnotesize,
    breakatwhitespace=false,
    breaklines=true,
    captionpos=b,
    keepspaces=true,
    numbers=left,
    numbersep=5pt,
    showspaces=false,
    showstringspaces=false,
    showtabs=false,
    tabsize=2,
    extendedchars=false
}
\lstset{style=pystyle}

% 自定義超連結的顏色
\hypersetup{
    colorlinks=true,
    linkcolor=blue,
    urlcolor=blue,
    linktoc=all,
    pdfborder={0 0 0} % 移除邊框
}

% ---------------------------------------------

\begin{document}

\chead{\normalsize \textbf{Homework 10-bar Truss Optimization}}  % 可調整大小
\lhead{\normalsize SoLab training}
\rhead{\normalsize R12522621 李京睿}
\cfoot{\thepage}

% \tableofcontents\thispagestyle{fancy} % 目錄顯示

% ---------------------------------------------

\section{Problem}

在特定的約束條件下,最佳化truss半徑,使10-bar Truss的整體重量達到最輕。\\\\
最佳化函數如下
\begin{equation*}
 	\min_{r_1, r_2} f(r_1, r_2) = \sum_{i=1}^{6}m_i(r_1) + \sum_{i=7}^{10} m_i(r_2)
\end{equation*}
約束條件如下
\begin{equation*}
	\begin{aligned}
		\left| \sigma_i \right| &<= \sigma_y \\
		\Delta s_2 &<= 0.02
	\end{aligned}
\end{equation*}
where
\begin{align*}
    f & : \text{所有桿件的質量} \\
    \Delta s_2 & : \text{node 2 的位移} \\
    \sigma_y & : \text{降伏應力} \\
    \sigma_i & : \text{所有桿件的應力}
\end{align*}

\subsection*{Solution}

使用講義提供之有限元素法,透過python執行計算後得到結果如下:
\begin{align*}
    f &= 2.124 \times 10^6 \, \text{kg}
\end{align*}
where
\begin{align*}
    r_1 & = 0.300 \, \text{m} \\
    r_2 & = 0.266 \, \text{m}
\end{align*}

\href{https://github.com/JerryLee0731/hw_10_Bars_Truss_Optimization}{點擊這裡}查看完整程式內容。


\begingroup  % 隱藏預設參考文獻標題
    \renewcommand{\section}[2]{}
    \bibliographystyle{ieeetr}
    \bibliography{ref.bib}
\endgroup

\end{document}